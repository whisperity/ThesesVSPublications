%% thesesvspublications.sty
%% Copyright 2020 Whisperity
%
% This work may be distributed and/or modified under the
% conditions of the LaTeX Project Public License, either version 1.3
% of this license or (at your option) any later version.
% The latest version of this license is in
%   http://www.latex-project.org/lppl.txt
% and version 1.3 or later is part of all distributions of LaTeX
% version 2005/12/01 or later.
%
% This work has the LPPL maintenance status `maintained'.
%
% The Current Maintainer of this work is Whisperity.
%
% This work consists of the files thesesvspublications.sty
% and thesesvspublications.tex.
%
\documentclass{article}
\usepackage{thesesvspublications} % Own package! :3

\usepackage{etoolbox}
\usepackage{hyperref}
    \hypersetup{
        pdftitle={The thesesvspublications Package},    % title
        pdfauthor={Whisperity},     % author
        pdfsubject={Thesis to Reference assocation matrix generator},   % subject of the document
        pdfnewwindow=true,      % links in new window
        hidelinks,              % disable link borders
        %colorlinks=false,       % false: boxed links; true: colored links
    }
\usepackage[capitalize]{cleveref}
\usepackage{minted}
    \usemintedstyle{borland}
    \setminted{autogobble}
\usepackage{placeins}
\usepackage{xcolor}

\newrobustcmd{\release}[3]{%
    \subsection{\emph{#1}: \texttt{#2} - #3}\label{release-#2}%
}

\newrobustcmd{\command}[3]{%
    \paragraph{\textsf{#1}} \textbf{\color{blue} \texttt{#2}}

    #3
}

\newrobustcmd{\tvpDisable}{%
    \let\theOldTVP\ThesesVSPublications%
    \def\ThesesVSPublications{}%
}
\newrobustcmd{\tvpEnable}{%
    \let\ThesesVSPublications\theOldTVP
}

% Escaped special characters for \texttt and the \write calls.
% (via http://tex.stackexchange.com/a/223089)
\def\atchar#1{}
\edef\atchar{\expandafter\atchar\string\@}
\def\bslchar#1{}
\edef\bslchar{\expandafter\bslchar\string\\}
\def\spchar#1{}
\edef\spchar{\expandafter\spchar\string\ }
\def\lbchar#1{}
\edef\lbchar{\expandafter\lbchar\string\{}
\def\rbchar#1{}
\edef\rbchar{\expandafter\rbchar\string\}}
\def\hschar#1{}
\edef\hschar{\expandafter\hschar\string\#}

\newwrite\tempfile

\begin{document}

\vspace*{0in}
\begin{center}
    \LARGE The \textsf{thesesvspublications} Package \\
    \small \rule{0in}{1em} Thesis to Reference association matrix generator \\
    \large \rule{0in}{2em} Whisperity \\
    % \\
    \rule{0em}{2em}\thesesvspublicationsReleaseDate \\
    Version \thesesvspublicationsVersion
\end{center}

\tableofcontents

\section{Introduction}\label{intro}
The \textsf{thesesvspublications} package allows the convenient creation of bibliography association matrices.
The package aims to automate the generation, without the user having to manually write the table's code and have to deal with the inefficiency of copy-pasting information.
The package's interface is simple, the user needs to define the table's contents (\cref{define}) and then can generate the table (\cref{tablegen}).

Note that it is important to have \emph{every} necessary thesis and publication defined \textbf{before} the first call to a table generation, otherwise, only a partial table will be generated.

\subsection{Association Matrix}\label{association-mx}
At the author's University, it is customary for Ph.D.\ students to write their doctoral dissertation organised into individual \emph{theses} (corresponding to chapters), and to provide a breakdown on which of their publications are ``proof'' behind the thesis (chapter).

Commonly, the dissertation's \emph{``Introduction''} and \emph{``Summary''} chapters will provide the full table.
In addition, at the end of each individual thesis chapters will provide a table with the current thesis's row in the table \emph{highlighted}.

An example usage and association matrix is provided in \cref{example}.

\subsection{Licence}
Copyright \textcopyright\ 2020 Whisperity.
Permission is granted to copy, distribute and\slash or modify this software under the terms of the \emph{\LaTeX{} Project Public License}, version \emph{1.3c} or newer.\footnote{%
    \url{http://www.latex-project.org/lppl}%
}

\subsection{Example}\label{example}
The code in \cref{assoc-mx-example-code} shows the usage of this package.
The definitions (\cref{define}) should be done early in the document, but at least before the first table render (\cref{tablegen}) call.

\immediate\openout\tempfile=assoc-mx-example.aux
    \immediate\write\tempfile{\bslchar definethesis{thesesvspublications}{Theses vs.\ Publications}}
    \immediate\write\tempfile{\bslchar definethesis{latex}{\bslchar LaTeX}}
    \immediate\write\tempfile{}
    \makeatletter
       \immediate\write\tempfile{\@percentchar\spchar Using the bibliography at the end of document!}
    \makeatother
    \immediate\write\tempfile{\bslchar definepublication{tvpdoc}{\bslchar cite{thesesvspublications}}}
    \immediate\write\tempfile{\bslchar definepublication{latexdoc}{\bslchar cite{latex}}}
    \immediate\write\tempfile{\bslchar definepublication{etoolboxdoc}{\bslchar cite{etoolbox}}}
    \immediate\write\tempfile{}
    \immediate\write\tempfile{\bslchar assignpublication{tvpdoc}{thesesvspublications}}
    \immediate\write\tempfile{\bslchar assignpublication{etoolboxdoc}{thesesvspublications}}
    \immediate\write\tempfile{\bslchar assignpublication{latexdoc}{latex}}
    \immediate\write\tempfile{\bslchar assignpublication{etoolboxdoc}{latex}}
    \immediate\write\tempfile{}
    \immediate\write\tempfile{\detokenize{\ThesesVSPublications}}
\immediate\closeout\tempfile

\begin{listing}[H]
    \inputminted{latex}{assoc-mx-example.aux}
    \caption{An example usage of \textsf{thesevspublications} defining a $2 \times 3$ matrix.}\label{assoc-mx-example-code}
\end{listing}

\tvpDisable
\input{assoc-mx-example.aux}
\tvpEnable

\begin{table}[H]
    \centering
    \ThesesVSPublications
    \caption{The output of the code in \cref{assoc-mx-example-code}, rendered into the \emph{assocation matrix}.}\label{assoc-mx-example}
\end{table}

\FloatBarrier
\section{Commands}\label{commands}
\subsection{Registering entries}\label{define}

\command{definethesis}{\bslchar definethesis\lbchar <key>\rbchar\lbchar <title>\rbchar\lbchar <summary>\rbchar}{%
Registers the thesis named \texttt{<key>} with the given \texttt{<title>}.
The \emph{title} of the thesis is printed as rows of the table, \textbf{in the order of} \texttt{\bslchar definethesis} calls.
}

\command{definepublication}{\bslchar definepublication\lbchar <key>\rbchar\lbchar <body>\rbchar}{%
Registers the publication named \texttt{<key>} with the given \texttt{<body>}.
The contents of \emph{body} is printed as the column headings of the table, \textbf{in the order of} \texttt{\bslchar definepublication} calls.
}

\command{assignpublication}{\bslchar assignpublication\lbchar <publication>\rbchar\lbchar <thesis>\rbchar}{%
Registers the \texttt{<publication>} to be relevant for the \texttt{<thesis>}.
The cells at the intersection of the thesis's row and the publication's column will have a \textbullet{}    , indicating the relevancy.
}

\subsection{Queries}\label{query}

\command{numtheses}{\bslchar numtheses}{%
Returns the number of theses defined.
}

\command{thesistitle}{\bslchar thesistitle\lbchar <key>\rbchar}{%
Returns the registered \emph{title} for the thesis registered with \texttt{<key>}.
}

\command{numpublications}{\bslchar numpublications}{%
Returns the number of publications defined.
}

\command{publication}{\bslchar publication\lbchar <key>\rbchar}{%
Returns the registered \emph{body} for the publication registered with \texttt{<key>}.
}

\subsection{Table generator: \texttt{ThesesVSPublications}}\label{tablegen}

\command{ThesesVSPublications}{\bslchar ThesesVSPublications[<thesis>]}{%
Generates the full table incorporating the previously registered data.
Internally, a \texttt{tabular} environment is created with the table's data generated by the package.
It is the user's responsibility to wrap this into a floating \texttt{table}, if they so wish.

If \texttt{[<thesis>]} is given a thesis's \emph{key}, the row for that thesis will be \textbf{highlighted}.
Highlighting is done by typesetting the thesis's name in \textbf{bold face}, and changing every \textbullet{} in the \emph{other thesis'} rows to \textopenbullet{}.

\begin{table}[H]
    \centering
    \ThesesVSPublications[latex]
    \caption{Using the definitions in \cref{assoc-mx-example-code}, but rendering a highlighted row (for thesis \emph{key} \texttt{latex}, via \texttt{\bslchar ThesesVSPublications[latex]}) in the table.}\label{highlight-row}
\end{table}
}

% Let's use an example of 2×2 matrix for the customisations.
\newcommand{\docExampleResetTableTwoByTwo}{%
    \tvpReset%
    \definethesis{sample}{Sample}%
    \definethesis{simple}{Simple}%
    \definepublication{x}{x}%
    \definepublication{y}{y}%
    \assignpublication{x}{sample}%
    \assignpublication{y}{simple}%
}
\newcommand{\loadAndRenderResettingExample}[2][0]{%
    \begin{figure}[H]
        \centering
        \inputminted{latex}{#2}

        \docExampleResetTableTwoByTwo
        \input{#2}
        \ThesesVSPublications[#1]
    \end{figure}%
}

\subsection{Customisation}\label{customise}
All aspects on how the individual elements in the table are rendered can be customised by setting \emph{renderer} control sequences.

\makeatletter
    \def\defaultTopCorner{\tvp@Renderer@Default@TopCorner}
\makeatother
\command{tvpSetTopCorner}{\bslchar tvpSetTopCorner\lbchar <new-heading>\rbchar}{%
    Sets the rendered table's top left cell value (by default, ``\emph{\defaultTopCorner}'') to the given \texttt{<new-heading>} value.

    \immediate\openout\tempfile=settopcorner-example.aux
        \immediate\write\tempfile{\bslchar tvpSetTopCorner{Publication matrix}}
    \immediate\closeout\tempfile
    \loadAndRenderResettingExample{settopcorner-example.aux}
}

\command{tvpSetPublicationHeading}{\bslchar tvpSetPublicationHeading\lbchar<render-command>\rbchar}{%
    Sets the rendered table's top row's publication cells to be rendered via passing the publication's body (see \cref{query}) to the given \texttt{<render-command>}.
    This command must be a command that takes \textbf{1} argument, defined by the user earlier.

    \immediate\openout\tempfile=setpublheading-example.aux
        \immediate\write\tempfile{\bslchar newcommand{\bslchar ttPublHead}[1]{\bslchar texttt{\hschar1}}}
        \immediate\write\tempfile{\bslchar tvpSetPublicationHeading{\bslchar ttPublHead}}
    \immediate\closeout\tempfile
    \loadAndRenderResettingExample{setpublheading-example.aux}
}

\command{tvpSetThesisName}{\bslchar tvpSetThesisName(Highlighted)\lbchar<render-command>\rbchar}{%
    Sets the rendered table's left cells' to be rendered via passing the thesis's index and title (see \cref{query}) to the given \texttt{<render-command>}.
    This command must be a command that takes \textbf{2} arguments, defined by the user earlier.
    The \texttt{Highlighted} version sets the renderer for the highlighted row, if highlighted rendering (see \cref{tablegen}) is done.

    \immediate\openout\tempfile=setthesisname-example.aux
        \immediate\write\tempfile{\bslchar newcommand{\bslchar parenThesis}[2]{(\hschar1) \bslchar emph{\hschar2}}}
        \immediate\write\tempfile{\bslchar newcommand{\bslchar parenThesisBl}[2]{!\hschar1! \bslchar texttt{\hschar2}}}
        \immediate\write\tempfile{\bslchar tvpSetThesisName{\bslchar parenThesis}}
        \immediate\write\tempfile{\bslchar tvpSetThesisNameHighlighted{\bslchar parenThesisBl}}
    \immediate\closeout\tempfile
    \loadAndRenderResettingExample[simple]{setthesisname-example.aux}
}

\makeatletter
    \def\defaultIndicator{\tvp@Renderer@Default@Indicator}
\makeatother
\command{tvpSetIndicator}{\bslchar tvpSetIndicator\lbchar<indicator>\rbchar}{%
    Sets the indicators of thesis-to-publication assignment in the matrix in \emph{non-highlighting} mode to the result of the \texttt{<indicator>}.
    The default indicator is: \defaultIndicator{}.

    \immediate\openout\tempfile=setindicator-example.aux
        \immediate\write\tempfile{\bslchar tvpSetIndicator{p}}
    \immediate\closeout\tempfile
    \loadAndRenderResettingExample{setindicator-example.aux}
}

\makeatletter
    \def\defaultIndicatorHC{}%\tvp@Renderer@Default@Indicator@Highlighted@Current}
    \def\defaultIndicatorHO{}%\tvp@Renderer@Default@Indicator@Highlighted@Other}
\makeatother
\command{tvpSetIndicatorHighlighted}{\linebreak\bslchar tvpSetIndicatorHighlighted\lbchar<hl-indicator>\rbchar\lbchar<ot-indicator>\rbchar}{%
    Sets the indicators of thesis-to-publication assignment in the matrix in \emph{highlighting} mode to the result of the \texttt{<hl-indicator>} for the \textbf{highlighted row} and \texttt{<ot-indicator>} for every other row.
    The default indicators are: \defaultIndicatorHC{} and \defaultIndicatorHO{}, respectively.

    \immediate\openout\tempfile=setindicator-hl-example.aux
        \immediate\write\tempfile{\bslchar tvpSetIndicatorHighlighted{c}{o}}
    \immediate\closeout\tempfile
    \loadAndRenderResettingExample[simple]{setindicator-hl-example.aux}
}

\FloatBarrier
\subsection{Multiple distinct tables: \texttt{tvpReset}}

\command{tvpReset}{\bslchar tvpReset}{%
Clears \textbf{all} internal data structures and customisations for the package, allowing the user to later typeset a different, independent matrix.
The previously defined entries are lost, and in case the ``previous'' matrix is needed, it must be set up again with a sequence of definition commands (see \cref{define}).

For example, putting the code in \cref{assoc-mx-example-code} and \cref{reset-example-code} in the same source file will create the smaller matrix (seen in \cref{reset-example}) at the second invocation of \texttt{\bslchar ThesesVSPublications}.

\immediate\openout\tempfile=reset-example.aux
    \immediate\write\tempfile{\bslchar tvpReset}
    \immediate\write\tempfile{\bslchar definethesis{sample}{Sample}}
    \immediate\write\tempfile{\bslchar definepublication{x}{x}}
    \immediate\write\tempfile{\bslchar assignpublication{x}{sample}}
    \immediate\write\tempfile{\detokenize{\ThesesVSPublications}}
\immediate\closeout\tempfile

\begin{listing}[H]
    \inputminted{latex}{reset-example.aux}
    \caption{Flushing the internal data structure so a different matrix can be typeset.}\label{reset-example-code}
\end{listing}

\tvpDisable
\input{reset-example.aux}
\tvpEnable

\begin{table}[H]
    \centering
    \ThesesVSPublications
    \caption{The output of the code in \cref{reset-example-code}.
        This should be a $1 \times 1$ matrix (excluding header row).}\label{reset-example}
\end{table}


\FloatBarrier
}

\section{Development}
The development of \textsf{thesesvspublications} is done on GitHub: \url{http://github.com/whisperity/thesesvspublications}.
Please report issues and submit patches here.

Note that this document acts as a \textbf{test file} to the package, not only a user-facing documentation.
Any new command or change should be supplemented with appropriate documentation, and the check of the rendered examples.
If there is a discrepancy or regression in the looks of \emph{this document} that is not explained by the changes, it should be investigated!

\section{Changelog}\label{changelog}
\release{2020/10/24}{v1.0}{Initial release}
\begin{itemize}
    \item Initial release of the package.
    \item Basic functionality of definition and assignment added.
    \item Configuration capabilities are added.
\end{itemize}


\begin{thebibliography}{3}
\bibitem{thesesvspublications}
Whisperity. \textit{The \textsf{Theses vs.\ Publications} Package -- Thesis to Reference association matrix generator}.
Online, \url{http://ctan.org/pkg/thesesvspublications} (accessed 2020/10/24), 2020.

\bibitem{latex} 
Leslie B.\ Lamport, Donald E.\ Knuth et al. 
\textit{\LaTeX{} -- A document preparation system}. 
Online, \url{http://latex-project.org/} (accessed 2020/10/24), 1984.

\bibitem{etoolbox} 
Philipp Lehman, Joseph Wright.
\textit{The \textsf{etoolbox} Package -- An e-\TeX{} Toolbox for Class and Package Authors}
Online, \url{http://ctan.org/pkg/etoolbox} (accessed 2020/10/24), 2007.
\end{thebibliography}

\end{document}